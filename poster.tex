\documentclass[a0paper, landscape, 25pt]{tikzposter}

\usepackage[T1]{fontenc}
\usepackage{graphicx}
\usepackage{times}
\usepackage{booktabs}
\usepackage{hyperref}
\usepackage{capt-of}
\usepackage{qrcode}
\usepackage{multirow}
\usepackage{setspace}
\usepackage{epstopdf}
\usetikzlibrary{arrows.meta, positioning, calc}
%!TEX root = poster.tex

\tikzposterlatexaffectionproofoff
\usetheme{Basic}

% colours are picked from the southampton.ac.uk weebsite
\definecolorstyle{Soton} {
    \definecolor{colorOne}{HTML}{002e3b}    % dark blue-ish
    \definecolor{colorTwo}{HTML}{005c84}    % blue-ish
    \definecolor{colorThree}{HTML}{fcbc00}  % orange-ish
}{
    % Background Colors
    \colorlet{backgroundcolor}{white}
    \colorlet{framecolor}{colorOne}
    % Title Colors
    \colorlet{titlefgcolor}{white}
    \colorlet{titlebgcolor}{colorOne}
    % Block Colors
    \colorlet{blocktitlebgcolor}{colorTwo}
    \colorlet{blocktitlefgcolor}{white}
    \colorlet{blockbodybgcolor}{white}
    \colorlet{blockbodyfgcolor}{black}
    % Innerblock Colors
    \colorlet{innerblocktitlebgcolor}{white}
    \colorlet{innerblocktitlefgcolor}{black}
    \colorlet{innerblockbodybgcolor}{colorTwo!20!white}
    \colorlet{innerblockbodyfgcolor}{black}
    % Note colors
    \colorlet{notefgcolor}{black}
    \colorlet{notebgcolor}{colorTwo!50!white}
    \colorlet{noteframecolor}{colorTwo}
}

% style the default blocks
\defineblockstyle{Block}{
    titlewidthscale=1, bodywidthscale=1, titleleft,
    titleoffsetx=0pt, titleoffsety=0pt, bodyoffsetx=0pt, bodyoffsety=0pt,
    bodyverticalshift=0pt, roundedcorners=0, linewidth=0pt, titleinnersep=1cm,
    bodyinnersep=1cm
}{
    \ifBlockHasTitle%
        \draw[draw=none, fill=blocktitlebgcolor]
           (blocktitle.south west) rectangle (blocktitle.north east);
    \fi%
    \draw[draw=none, fill=blockbodybgcolor] %
        (blockbody.north west) [rounded corners=0] -- (blockbody.south west) --
        (blockbody.south east) [rounded corners=0]-- (blockbody.north east) -- cycle;
}

% style the TL;DR block
\defineblockstyle{Emphasis}{
    titlewidthscale=1, bodywidthscale=1, titleleft,
    bodyverticalshift=0pt, roundedcorners=0, linewidth=0pt, titleinnersep=1cm,
    bodyinnersep=1cm
}{
    \ifBlockHasTitle%
        \draw[draw=colorTwo, fill=blocktitlebgcolor]
           (blocktitle.south west) rectangle (blocktitle.north east);
    \fi%
    \draw[draw=none, fill=colorTwo] %
        (blockbody.north west) [rounded corners=0] -- (blockbody.south west) --
        (blockbody.south east) [rounded corners=0]-- (blockbody.north east) -- cycle;
}

% very simple title style
\definetitlestyle{Simple}{
    width=\paperwidth,
    titletotopverticalspace=0mm,
    titletoblockverticalspace=20mm,
    innersep=1cm,
}{
    \draw[draw=none, fill=titlebgcolor]%
    (\titleposleft,\titleposbottom) rectangle (\titleposright,\titlepostop); %
}

\usecolorstyle{Soton}
\usetitlestyle{Simple}
\useblockstyle{Block}
\useinnerblockstyle[roundedcorners=0]{Default}

% uncomment for long titles
% % https://tex.stackexchange.com/questions/180234/how-can-i-make-my-title-wrap-in-a-tikzposter
% \makeatletter
% \def\title#1{\gdef\@title{\scalebox{\TP@titletextscale}{%
%     \begin{minipage}[t]{0.4\linewidth}
%         #1
%         \par
%         \end{minipage}%
%     }}}
% \makeatother

% fix tikzlibrary{backgrounds}: https://tex.stackexchange.com/questions/236645/tikzposter-does-not-work-with-pgfplots/236672
\pgfdeclarelayer{background}
\pgfsetlayers{backgroundlayer,background,main,notelayer}


% footnotes: https://tex.stackexchange.com/questions/350010/how-to-write-a-footnote-in-tikz-poster-class/350018
\let\thempfootnote\thefootnote%http://tex.stackexchange.com/questions/956/footnotemark-and-footnotetext-in-minipage#959
\newcommand\printfootnote[1]{% to get different numbers for different footnotes
\addtocounter{footnote}{1}%
\footnotetext{#1}}

\input{math_commands}
%!TEX root = poster.tex

\usepackage{tikz}
\usepackage{xcolor}
\usepackage{gensymb}

\usetikzlibrary{shapes, backgrounds, fit, calc, positioning, arrows.meta, bending}

% colorbrewer colours, taken from https://github.com/vtraag/tikz-colorbrewer/blob/master/colorbrewer.sty because I couldn't get \usetikzlibrary{colorbrewer} to work
\definecolor{RdYlBu-4-1}{RGB}{215,25,28}
\definecolor{RdYlBu-4-2}{RGB}{253,174,97}
\definecolor{RdYlBu-4-3}{RGB}{171,217,233}
\definecolor{RdYlBu-4-4}{RGB}{44,123,182}
\definecolor{Blues-4-1}{RGB}{239,243,255}
\definecolor{Blues-4-2}{RGB}{189,215,231}
\definecolor{Blues-4-3}{RGB}{107,174,214}
\definecolor{Blues-4-4}{RGB}{33,113,181}
\definecolor{Oranges-4-1}{RGB}{254,237,222}
\definecolor{Oranges-4-2}{RGB}{253,190,133}
\definecolor{Oranges-4-3}{RGB}{253,141,60}
\definecolor{Oranges-4-4}{RGB}{217,71,1}
\definecolor{Purples-4-1}{RGB}{242,240,247}
\definecolor{Purples-4-2}{RGB}{203,201,226}
\definecolor{Purples-4-3}{RGB}{158,154,200}
\definecolor{Purples-4-4}{RGB}{106,81,163}

\newcommand{\AxisRotator}[1][rotate=0]{%
    \tikz [x=1.1mm,y=1.1mm,line width=.1ex,>={Latex[length=.3em, bend]},#1] \draw (0,0) arc (-150:170:1);%
}

% https://tex.stackexchange.com/a/66220
\def\centerarc[#1](#2)(#3:#4:#5)% Syntax: [draw options] (center) (initial angle:final angle:radius)
    { \draw[#1] ($(#2)+({#5*cos(#3)},{#5*sin(#3)})$) arc (#3:#4:#5); }

\tikzset{
    every picture/.append style={x=5mm, y=5mm},
    circ/.style={circle, draw=black, fill=black, minimum size=2.5mm, inner sep=0mm},
    box/.style={rectangle,draw=black,thick, minimum size=5mm},
    object/.style={draw, dotted, inner sep=0.3em, fill=black!3, rounded corners=1mm},
    desc/.style={rectangle},
    sedge/.style={->, >=stealth},
    dedge/.style={<->, >=stealth},
    varname/.style={midway, above, opacity=1},
    mnist1/.pic={
        % images
        \node[inner sep=0] (input) at (0, 0) {\includegraphics[scale=0.7, trim={3.5mm 3.5mm 3.5mm 3.5mm}, clip]{resources/mnist-32-11.pdf}};
        \node[inner sep=0] (target) at (20.5, 0) {\includegraphics[scale=0.7, trim={3.5mm 3.5mm 3.5mm 3.5mm}, clip]{resources/mnist-32-11.pdf}};
        \node[inner sep=0] (y0) at (0, -4) {\includegraphics[scale=0.7, trim={3.5mm 3.5mm 3.5mm 3.5mm}, clip]{resources/mnist-32-0.pdf}};
        \node[inner sep=0] (y1) at (8, -4) {\includegraphics[scale=0.7, trim={3.5mm 3.5mm 3.5mm 3.5mm}, clip]{resources/mnist-32-1.pdf}};
        \node[inner sep=0] (y2) at (16, -4) {\includegraphics[scale=0.7, trim={3.5mm 3.5mm 3.5mm 3.5mm}, clip]{resources/mnist-32-2.pdf}};
        \node[inner sep=0] (y10) at (20.5, -4) {\includegraphics[scale=0.7, trim={3.5mm 3.5mm 3.5mm 3.5mm}, clip]{resources/mnist-32-10.pdf}};

        % feature vectors
        % input fv
        \node (fvinput bottom)[box, scale=0.36, fill=Oranges-4-3] at (4, -0.5) {};
        \node (fvinput) [box, scale=0.36, fill=Blues-4-4] at (4, 0) {};
        \node [box, scale=0.36, fill=Blues-4-4] at (4, 0.5) {};

        \node (fvinput2 bottom)[box, scale=0.36, fill=Oranges-4-3] at (12, -0.5) {};
        \node (fvinput2) [box, scale=0.36, fill=Blues-4-4] at (12, 0) {};
        \node [box, scale=0.36, fill=Blues-4-4] at (12, 0.5) {};
        % fv0
        \node (fv0 top) [box, scale=0.36, fill=Oranges-4-1] at (4, -3.5) {};
        \node (fv0) [box, scale=0.36, fill=Purples-4-1] at (4, -4) {};
        \node [box, scale=0.36, fill=Oranges-4-1] at (4, -4.5) {};
        % fv1
        \node (fv1 top) [box, scale=0.36, fill=Blues-4-3!50!Blues-4-4] at (12, -3.5) {};
        \node (fv1) [box, scale=0.36, fill=Blues-4-3] at (12, -4) {};
        \node [box, scale=0.36, fill=Oranges-4-2] at (12, -4.5) {};

        % bendy connection between fvs
        \draw [sedge, in=200, out=-20] (y0.south east) to (y1.south west);
        \draw [sedge, in=200, out=-20] (y1.south east) to (y2.south west);
        
        % MSEs
        \draw [ultra thick, color=colorTwo] (fvinput bottom) -- (fv0 top) node (mse0) [midway, scale=0.2, fill=white] {MSE};
        \draw [ultra thick, color=colorTwo] (fvinput2 bottom) -- (fv1 top) node (mse1) [midway, scale=0.2, fill=white] {MSE};

        % input encoder
        \draw [sedge] (input) -- (fvinput) node [midway, scale=0.2, trapezium, rotate=-90, trapezium angle=70, inner ysep=12mm, draw, fill=colorThree!10!white] {Encoder};
        % y0 encoder
        \draw [sedge] (y0) -- (fv0) node [midway, scale=0.2, trapezium, rotate=-90, trapezium angle=70, inner ysep=12mm, draw, fill=colorThree!10!white] {Encoder};
        % y1 encoder
        \draw [sedge] (y1) -- (fv1) node [midway, scale=0.2, trapezium, rotate=-90, trapezium angle=70, inner ysep=12mm, draw, fill=colorThree!10!white] {Encoder};

        % backprop
        \draw [sedge, color=colorTwo] (mse0) -- (y1.north west)
        node [midway, scale=0.2, sloped, above] {$- \partial \text{ MSE}$}
        node [midway, scale=0.2, sloped, below] {$\partial \text{ Step 0}$};
        \draw [sedge, color=colorTwo] (mse1) -- (y2.north west)
        node [midway, scale=0.2, sloped, above] {$- \partial \text{ MSE}$}
        node [midway, scale=0.2, sloped, below] {$\partial \text{ Step 1}$};

        \node [scale=0.2, below = 0mm of y0] {Step 0};
        \node [scale=0.2, below = 0mm of y1] {Step 1};
        \node [scale=0.2, below = 0mm of y2] {Step 2};                        
        \node [scale=0.2, below = 0mm of y10] {Step 10};
        \node [scale=0.2, above = 0mm of input] {Input};
        \node [scale=0.2, above = 0mm of target] {Target};

        \draw [ultra thick, color=colorThree!50!black] (y10) -- (target) node [midway, fill=white, scale=0.2] {set loss};

        \node [scale=0.4] at ($(y2)!0.5!(y10)$) {\ldots};
    },
    mnist2/.pic = {
        % images
        \node[inner sep=0] (input) at (0, 0) {\includegraphics[scale=0.7, trim={3.5mm 3.5mm 3.5mm 3.5mm}, clip]{resources/mnist-121-11.pdf}};
        \node[inner sep=0] (target) at (20.5, 0) {\includegraphics[scale=0.7, trim={3.5mm 3.5mm 3.5mm 3.5mm}, clip]{resources/mnist-121-11.pdf}};
        \node[inner sep=0] (y0) at (0, -4) {\includegraphics[scale=0.7, trim={3.5mm 3.5mm 3.5mm 3.5mm}, clip]{resources/mnist-121-0.pdf}};
        \node[inner sep=0] (y1) at (8, -4) {\includegraphics[scale=0.7, trim={3.5mm 3.5mm 3.5mm 3.5mm}, clip]{resources/mnist-121-1.pdf}};
        \node[inner sep=0] (y2) at (16, -4) {\includegraphics[scale=0.7, trim={3.5mm 3.5mm 3.5mm 3.5mm}, clip]{resources/mnist-121-2.pdf}};
        \node[inner sep=0] (y10) at (20.5, -4) {\includegraphics[scale=0.7, trim={3.5mm 3.5mm 3.5mm 3.5mm}, clip]{resources/mnist-121-10.pdf}};

        % feature vectors
        % input fv
        \node (fvinput bottom)[box, scale=0.36, fill=Purples-4-3] at (4, -0.5) {};
        \node (fvinput) [box, scale=0.36, fill=Oranges-4-4] at (4, 0) {};
        \node [box, scale=0.36, fill=Oranges-4-3] at (4, 0.5) {};

        \node (fvinput2 bottom)[box, scale=0.36, fill=Purples-4-3] at (12, -0.5) {};
        \node (fvinput2) [box, scale=0.36, fill=Oranges-4-4] at (12, 0) {};
        \node [box, scale=0.36, fill=Oranges-4-3] at (12, 0.5) {};
        % fv0
        \node (fv0 top) [box, scale=0.36, fill=Oranges-4-1] at (4, -3.5) {};
        \node (fv0) [box, scale=0.36, fill=Purples-4-1] at (4, -4) {};
        \node [box, scale=0.36, fill=Oranges-4-1] at (4, -4.5) {};
        % fv1
        \node (fv1 top) [box, scale=0.36, fill=Oranges-4-3] at (12, -3.5) {};
        \node (fv1) [box, scale=0.36, fill=Oranges-4-3] at (12, -4) {};
        \node [box, scale=0.36, fill=Blues-4-2] at (12, -4.5) {};

        % bendy connection between fvs
        \draw [sedge, in=200, out=-20] (y0.south east) to (y1.south west);
        \draw [sedge, in=200, out=-20] (y1.south east) to (y2.south west);
        
        % MSEs
        \draw [ultra thick, color=colorTwo] (fvinput bottom) -- (fv0 top) node (mse0) [midway, scale=0.2, fill=white] {MSE};
        \draw [ultra thick, color=colorTwo] (fvinput2 bottom) -- (fv1 top) node (mse1) [midway, scale=0.2, fill=white] {MSE};

        % input encoder
        \draw [sedge] (input) -- (fvinput) node [midway, scale=0.2, trapezium, rotate=-90, trapezium angle=70, inner ysep=12mm, draw, fill=colorThree!10!white] {Encoder};
        % y0 encoder
        \draw [sedge] (y0) -- (fv0) node [midway, scale=0.2, trapezium, rotate=-90, trapezium angle=70, inner ysep=12mm, draw, fill=colorThree!10!white] {Encoder};
        % y1 encoder
        \draw [sedge] (y1) -- (fv1) node [midway, scale=0.2, trapezium, rotate=-90, trapezium angle=70, inner ysep=12mm, draw, fill=colorThree!10!white] {Encoder};

        % backprop
        \draw [sedge, color=colorTwo] (mse0) -- (y1.north west)
        node [midway, scale=0.2, sloped, above] {$- \partial \text{ MSE}$}
        node [midway, scale=0.2, sloped, below] {$\partial \text{ Step 0}$};
        \draw [sedge, color=colorTwo] (mse1) -- (y2.north west)
        node [midway, scale=0.2, sloped, above] {$- \partial \text{ MSE}$}
        node [midway, scale=0.2, sloped, below] {$\partial \text{ Step 1}$};

        \node [scale=0.2, below = 0mm of y0] {Step 0};
        \node [scale=0.2, below = 0mm of y1] {Step 1};
        \node [scale=0.2, below = 0mm of y2] {Step 2};                        
        \node [scale=0.2, below = 0mm of y10] {Step 10};
        \node [scale=0.2, above = 0mm of input] {Input};
        \node [scale=0.2, above = 0mm of target] {Target};

        \draw [ultra thick, color=colorThree!50!black] (y10) -- (target) node [midway, fill=white, scale=0.2] {set loss};

        \node [scale=0.4] at ($(y2)!0.5!(y10)$) {\ldots};
    },
    clevr1/.pic = {
        % images
        \node[inner sep=0] (input) at (0, 1) {\includegraphics[scale=0.7, trim={3.5mm 3.5mm 3.5mm 3.5mm}, clip]{resources/clevr-23.pdf}};
        \node[inner sep=0] (target) at (18, 1) {\includegraphics[scale=0.7, trim={3.5mm 3.5mm 3.5mm 3.5mm}, clip]{resources/clevr-23--2.pdf}};
        \node[inner sep=0] (y0) at (0, -4) {\includegraphics[scale=0.7, trim={3.5mm 3.5mm 3.5mm 3.5mm}, clip]{resources/clevr-23-0.pdf}};
        \node[inner sep=0] (y1) at (11, -4) {\includegraphics[scale=0.7, trim={3.5mm 3.5mm 3.5mm 3.5mm}, clip]{resources/clevr-23-1.pdf}};
        \node[inner sep=0] (y10) at (18, -4) {\includegraphics[scale=0.7, trim={3.5mm 3.5mm 3.5mm 3.5mm}, clip]{resources/clevr-23-10.pdf}};

        % feature vectors
        % input fv
        \node (fvinput bottom)[box, scale=0.36, fill=Purples-4-4] at (6, 0.5) {};
        \node (fvinput) [box, scale=0.36, fill=Blues-4-4] at (6, 1) {};
        \node (fvinput top) [box, scale=0.36, fill=Oranges-4-3] at (6, 1.5) {};
        \node [above = 0mm of fvinput top, scale=0.2] {512d};

        \node (fvtarget bottom)[box, scale=0.36, fill=Purples-4-4] at (11, 0.5) {};
        \node (fvtarget) [box, scale=0.36, fill=Blues-4-4] at (11, 1) {};
        \node [box, scale=0.36, fill=Oranges-4-3] at (11, 1.5) {};
        % fv0
        \node (fv0 top) [box, scale=0.36, fill=Oranges-4-1] at (6, -3.5) {};
        \node (fv0) [box, scale=0.36, fill=Blues-4-1] at (6, -4) {};
        \node [box, scale=0.36, fill=Purples-4-1] at (6, -4.5) {};

        % bendy connection between fvs
        \draw [sedge, in=200, out=-20] (y0.south east) to (y1.south west);

        % MSEs
        \draw [ultra thick, color=colorTwo] (fvinput bottom) -- (fv0 top) node (mse0) [midway, scale=0.2, fill=white] {MSE};
        \draw [ultra thick, color=colorThree!50!black] (fvtarget) -- (fvinput) node [midway, scale=0.2, fill=white] {MSE loss};

        % input encoder
        \draw [sedge] (input) -- (fvinput) node [midway, scale=0.2, trapezium, rotate=-90, trapezium angle=70, inner ysep=22mm, draw, fill=colorThree!10!white] {\LARGE ResNet34};
        % y0 encoder
        \draw [sedge] (y0) -- (fv0) node [midway, scale=0.2, trapezium, rotate=-90, trapezium angle=70, inner ysep=12mm, draw, fill=colorThree!10!white] {Encoder};
        % target encoder
        \draw [sedge] (target) -- (fvtarget) node [midway, scale=0.2, trapezium, rotate=90, trapezium angle=70, inner ysep=12mm, draw, fill=colorThree!10!white] {Encoder};

        % backprop
        \draw [sedge, color=colorTwo] (mse0) -- (y1.north west)
        node [midway, scale=0.2, sloped, above] {$- \partial \text{ MSE}$}
        node [midway, scale=0.2, sloped, below] {$\partial \text{ Step 0}$};

        \node [scale=0.2, below = 0mm of y0] {Step 0};
        \node [scale=0.2, below = 0mm of y1] {Step 1};
        \node [scale=0.2, below = 0mm of y10] {Step 10};
        \node [scale=0.2, above = 0mm of input] {Input};
        \node [scale=0.2, above = 0mm of target] {Target};

        \draw [ultra thick, color=colorThree!50!black] (y10) -- (target) node [midway, fill=white, scale=0.2] {set loss};

        \node [scale=0.4] at ($(y1)!0.5!(y10)$) {\ldots};
    },
    symmetry/.pic = {
        % \node [scale=0.9] at (6.5, 2) {The Responsibility Problem};

        \pic at (0, 0) {square};
        \begin{scope}[on background layer]
            \node (s1) [object, fit=(a) (b) (c) (d)] {};
        \end{scope}

        \pic at (13, 0) {square};
        \begin{scope}[on background layer]
            \node (s4) [object, fit=(a) (b) (c) (d)] {};
        \end{scope}

        \pic [rotate=-30] at (4, -3) {square};
        \begin{scope}[on background layer]
            \node (s2) [object, fit=(a) (b) (c) (d)] {};
        \end{scope}

        \pic [rotate=-30+90] at (9, -3) {square};
        \begin{scope}[on background layer]
            \node (s3) [object, fit=(a) (b) (c) (d)] {};
        \end{scope}

        \node [above = 0mm of s1, scale=0.7, color=colorTwo] {(a)};
        \node [above = 0mm of s2, scale=0.7, color=colorTwo] {(c)};
        \node [above = 0mm of s3, scale=0.7, color=colorTwo] {(d)};
        \node [above = 0mm of s4, scale=0.7, color=colorTwo] {(b)};

        \draw [sedge] (s1) -- (s4) node [above, midway] {\raisebox{-0.3mm}{\AxisRotator[xscale=-1]} $90\degree$};
        \draw [sedge, draw=red] (s2) -- (s3) node [below, midway] {\raisebox{-0.5mm}{\AxisRotator[xscale=-1]} $\epsilon$} node [above, midway, red, scale=0.4] {discontinuity};
        \draw [sedge] (s1.south) to[bend right] node [below=1.5mm, midway] {\raisebox{-0.5mm}{\AxisRotator[xscale=-1]} $30\degree$} (s2.west);
        \draw [sedge, text width=1.8cm] (s3.east) to[bend right] node [below=4.5mm, pos=0.8] {\raisebox{-0.5mm}{\AxisRotator[xscale=-1]} $60\degree - \epsilon$} (s4.south);
    },
    square/.pic = {
        \node (a) [circ, fill=RdYlBu-4-1] at (-0.8, 0.8) {};
        \node (b) [circ, fill=RdYlBu-4-2] at (0.8, 0.8) {};
        \node (c) [circ, fill=RdYlBu-4-3] at (0.8, -0.8) {};
        \node (d) [circ, fill=RdYlBu-4-4] at (-0.8, -0.8) {};
    },
    square2/.pic = {
        \node (a) [circ, fill=colorOne] at (-0.8, 0.8) {};
        \node (b) [circ, fill=colorOne] at (0.8, 0.8) {};
        \node (c) [circ, fill=colorOne] at (0.8, -0.8) {};
        \node (d) [circ, fill=colorOne] at (-0.8, -0.8) {};
    },
    model/.pic = {
        \pic (inputs) at (0, 0) {inputs};
        \pic (sort) at (8, 0) {sorted};
        \pic [scale=1.25, every node/.style={transform shape}] (output) at (12.75, 0) {output};
        \pic (dweights) at (16, 0) {dweights};
        \pic (cweights) at (20.5, 0) {cweights};

        \node at (2.5, 1.5) {$\mX$};

        \node at (6.35, -1.2) {\scriptsize sort};
        \draw [sedge] (5.6,  0.0) -- (7.2,  0.0);
        \draw [sedge] (5.6,  0.7) -- (7.2,  0.7);
        \draw [sedge] (5.6, -0.7) -- (7.2, -0.7);

        \node at (8.75, 1.5) {$\mvX$};

        \draw [sedge] (10.2,  0.0) -- (12,  0.0);
        \draw [sedge] (10.2,  0.7) -- (12,  0.7);
        \draw [sedge] (10.2, -0.7) -- (12, -0.7);

        \node at (12.75, 1.4) {$\vy$};
        \node at (12.75, -1.35) {\scriptsize dot product};

        \draw [sedge] (15.2,  0.0) -- (13.5,  0.0);
        \draw [sedge] (15.2,  0.7) -- (13.5,  0.7);
        \draw [sedge] (15.2, -0.7) -- (13.5, -0.7);

        \node at (17.0, 1.4) {\scriptsize $f([\scriptscriptstyle 0, \frac{1}{3}, \frac{2}{3}, 1 \displaystyle], \mbW)$};

        \node at (19.2, -1.2) {\scriptsize discretise};

        \node at (23, 1.4) {\scriptsize $f(\cdot, \mbW)$};

        \draw [sedge] (20,  0.0) -- (18.2,  0.0);
        \draw [sedge] (20,  0.7) -- (18.2,  0.7);
        \draw [sedge] (20, -0.7) -- (18.2, -0.7);
    },
    sorted/.pic = {
        \node [box, scale=0.5, fill=Blues-4-4] at (0.0, 0.7) {};
        \node [box, scale=0.5, fill=Blues-4-3] at (0.5, 0.7) {};
        \node [box, scale=0.5, fill=Blues-4-2] at (1.0, 0.7) {};
        \node [box, scale=0.5, fill=Blues-4-1] at (1.5, 0.7) {};

        \node [box, scale=0.5, fill=Oranges-4-4] at (0.0, 0.0) {};
        \node [box, scale=0.5, fill=Oranges-4-3] at (0.5, 0.0) {};
        \node [box, scale=0.5, fill=Oranges-4-2] at (1.0, 0.0) {};
        \node [box, scale=0.5, fill=Oranges-4-1] at (1.5, 0.0) {};

        \node [box, scale=0.5, fill=Purples-4-4] at (0.0, -0.7) {};
        \node [box, scale=0.5, fill=Purples-4-3] at (0.5, -0.7) {};
        \node [box, scale=0.5, fill=Purples-4-2] at (1.0, -0.7) {};
        \node [box, scale=0.5, fill=Purples-4-1] at (1.5, -0.7) {};
    },
    inputs/.pic = {
        \node at (0, 0) {$\Bigg\{$};
        \node at (5, 0) {$\Bigg\}$};

        \node [box, scale=0.625, fill=Blues-4-3] at (1, 0.625) {};
        \node [box, scale=0.625, fill=Oranges-4-4] at (1, 0.0) {};
        \node [box, scale=0.625, fill=Purples-4-3] at (1, -0.625) {};

        \node [box, scale=0.625, fill=Blues-4-2] at (2, 0.625) {};
        \node [box, scale=0.625, fill=Oranges-4-2] at (2, 0.0) {};
        \node [box, scale=0.625, fill=Purples-4-1] at (2, -0.625) {};

        \node [box, scale=0.625, fill=Blues-4-1] at (3, 0.625) {};
        \node [box, scale=0.625, fill=Oranges-4-3] at (3, 0.0) {};
        \node [box, scale=0.625, fill=Purples-4-4] at (3, -0.625) {};

        \node [box, scale=0.625, fill=Blues-4-4] at (4, 0.625) {};
        \node [box, scale=0.625, fill=Oranges-4-1] at (4, 0.0) {};
        \node [box, scale=0.625, fill=Purples-4-2] at (4, -0.625) {};
    },
    dweights/.pic = {
        \node [box, scale=0.5, fill=Blues-4-3] at (0.0, 0.7) {};
        \node [box, scale=0.5, fill=Blues-4-3] at (0.5, 0.7) {};
        \node [box, scale=0.5, fill=Blues-4-3] at (1.0, 0.7) {};
        \node [box, scale=0.5, fill=Blues-4-3] at (1.5, 0.7) {};

        \node [box, scale=0.5, fill=Oranges-4-2] at (0.0, 0.0) {};
        \node [box, scale=0.5, fill=Oranges-4-3] at (0.5, 0.0) {};
        \node [box, scale=0.5, fill=Oranges-4-3] at (1.0, 0.0) {};
        \node [box, scale=0.5, fill=Oranges-4-2] at (1.5, 0.0) {};

        \node [box, scale=0.5, fill=Purples-4-1] at (0.0, -0.7) {};
        \node [box, scale=0.5, fill=Purples-4-1] at (0.5, -0.7) {};
        \node [box, scale=0.5, fill=Purples-4-3] at (1.0, -0.7) {};
        \node [box, scale=0.5, fill=Purples-4-4] at (1.5, -0.7) {};
    },
    cweights/.pic = {
        \draw [very thick] (0, 0) -- (5, 0);
        \draw [very thick, sedge] (0, -1) -- (0, 1.4);
        \draw [very thick] (5, -1) -- (5, 1.4);

        \draw [thick, draw=Blues-4-3] (0, 0.5) -- (2.5, 0.5) -- (5, 0.5);
        \draw [thick, draw=Oranges-4-3] (0, 0) -- (2.5, 1) -- (5, 0);
        \draw [thick, draw=Purples-4-3] (0, -0.5) -- (2.5, -0.5) -- (5, 1);

        \node [circle, scale=0.25, fill=Oranges-4-4] at (0, 0) {};
        \node [circle, scale=0.25, fill=Blues-4-4] at (0, 0.5) {};
        \node [circle, scale=0.25, fill=Purples-4-4] at (0, -0.5) {};
        \node [circle, scale=0.25, fill=Oranges-4-4] at (1.667, 0.666) {};
        \node [circle, scale=0.25, fill=Blues-4-4] at (1.667, 0.5) {};
        \node [circle, scale=0.25, fill=Purples-4-4] at (1.667, -0.5) {};
        \node [circle, scale=0.25, fill=Oranges-4-4] at (3.333, 0.666) {};
        \node [circle, scale=0.25, fill=Blues-4-4] at (3.333, 0.5) {};
        \node [circle, scale=0.25, fill=Purples-4-4] at (3.333, 0) {};
        \node [circle, scale=0.25, fill=Purples-4-4] at (5, 1) {};
        \node [circle, scale=0.25, fill=Blues-4-4] at (5, 0.5) {};
        \node [circle, scale=0.25, fill=Oranges-4-4] at (5, 0) {};
    },
    output/.pic = {
        \node [box, scale=0.5, fill=Blues-4-3!50!Blues-4-2] at (0, 0.5) {};
        \node [box, scale=0.5, fill=Oranges-4-3!50!Oranges-4-2] at (0, 0.0) {};
        \node [box, scale=0.5, fill=Purples-4-1] at (0, -0.5) {};
    },
    ae/.pic = {
        \pic at (-3, 0) {square2};
        \begin{scope}[on background layer]
            \node (s1) [object, fit=(a) (b) (c) (d)] {};
        \end{scope}

        \node (enc) [trapezium, draw, rotate=-90, inner ysep=3.5mm, trapezium angle=70, fill=colorThree!10] at (3.5, 0) {Encoder};
        \node [box, scale=0.5, fill=Blues-4-3] at (5.5, 0.5) {};
        \node (fv) [box, scale=0.5, fill=Oranges-4-3] at (5.5, 0) {};
        \node [box, scale=0.5, fill=Purples-4-3] at (5.5, -0.5) {};
        \node (dec) [trapezium, draw, rotate=90, inner ysep=3.5mm, trapezium angle=70, fill=colorThree!10] at (7.5, 0) {~~~~MLP~~~~};


        \node [color=colorOne] (i4) at (0,  1.5) {(-1, -1)};
        \node [color=colorOne] (i3) at (0,  0.5) {(~1, -1)};
        \node [color=colorOne] (i2) at (0, -0.5) {(~1, ~1)};
        \node [color=colorOne] (i1) at (0, -1.5) {(-1, ~1)};

        \node [color=RdYlBu-4-1] (o4) at (11,  1.5) {(-1, ~1)};
        \node [color=RdYlBu-4-2] (o3) at (11,  0.5) {(~1, ~1)};
        \node [color=RdYlBu-4-3] (o2) at (11, -0.5) {(~1, -1)};
        \node [color=RdYlBu-4-4] (o1) at (11, -1.5) {(-1, -1)};

        \pic at (14, 0) {square};
        \begin{scope}[on background layer]
            \node (s2) [object, fit=(a) (b) (c) (d)] {};
        \end{scope}


        \draw [sedge] (i1) -- (enc.south |- i1);
        \draw [sedge] (i2) -- (enc.south |- i2);
        \draw [sedge] (i3) -- (enc.south |- i3);
        \draw [sedge] (i4) -- (enc.south |- i4);
        \draw [sedge] (enc) -- (fv);
        \draw [sedge] (fv) -- (dec);
        \draw [sedge] (dec.south |- o1) -- (o1);
        \draw [sedge] (dec.south |- o2) -- (o2);
        \draw [sedge] (dec.south |- o3) -- (o3);
        \draw [sedge] (dec.south |- o4) -- (o4);

    }
}


\renewcommand{\familydefault}{\sfdefault}
\usepackage{DejaVuSansMono}
\usepackage{FiraSans}


\title{Deep Set Prediction Networks}
\author{%
    \textbf{Yan Zhang},
    Jonathon Hare,
    Adam Pr\"ugel-Bennett
}
\institute{University of Southampton}
\titlegraphic{\includegraphics[scale=2.97]{logowhite}}


\settitle{
    \hspace*{1cm}
    \textcolor{white}{\fontsize{90}{90} \sc \@title}
    \hspace*{8cm}
    \textcolor{white}{\LARGE \@author}
    \hspace*{8cm}
    \raisebox{-1cm}{\@titlegraphic}
}

\begin{document}
    \maketitle


    \begin{columns}



        % left column
        \column{0.333333333333}
      
        \block{Sets are unordered collections of things}{
            \begin{itemize}
                \item Many things can be described as \textbf{sets of feature vectors}:

                \begin{itemize}
                    \item the set of objects in an image,
                    \item the set of points in a point cloud,
                    \item the set of nodes and edges in a graph,
                    \item the set of people reading this poster.
                \end{itemize}

                \item Predicting sets means object detection, molecule generation, etc.
                \item This paper is about doing this \textbf{vector-to-set} mapping properly.
                \item Compared to normal object detection methods:
                \begin{itemize}
                    \item Anchor-free, fully end-to-end, no post-processing.
                \end{itemize}
            \end{itemize}
        }
      
        \block{MLPs are not suited for sets}{
            \begin{itemize}
                \item Sets are \textbf{unordered}, but MLP and RNN outputs are \textbf{ordered}.
                \begin{itemize}
                    \item[$\rightarrow$] \textbf{Discontinuities} from \emph{responsibility problem}.
                \end{itemize}
                \item Let's look at a normal set auto-encoder:
            \end{itemize}

            \vspace{0.5cm}
            \begin{center}
                \scalebox{3.3}{
                    \begin{tikzpicture}
                        \fontsize{10}{10}\selectfont
                        \pic {ae};
                    \end{tikzpicture}
                }
            \end{center}

            \vspace{0.5cm}
            \begin{itemize}
                \item The responsibility problem:
            \end{itemize}
            \begin{center}
                \hspace*{5mm}
                \scalebox{4.0}{
                    \begin{tikzpicture}
                        \fontsize{10}{10}\selectfont
                        \pic {symmetry};
                    \end{tikzpicture}
                }
            \end{center}
            \vspace{0.5cm}

            \begin{itemize}
                \item \textcolor{colorTwo}{(a)} and \textcolor{colorTwo}{(b)} are the same set.
                \begin{itemize}
                    \item[$\rightarrow$] \textcolor{colorTwo}{(a)} and \textcolor{colorTwo}{(b)} encode to the same vector.
                    \item[$\rightarrow$] \textcolor{colorTwo}{(a)} and \textcolor{colorTwo}{(b)} have the same MLP output.
                \end{itemize}
                \item \textcolor{colorTwo}{(a)} is turned into \textcolor{colorTwo}{(b)} by rotating 90$\degree$.
                \begin{itemize}
                    \item[$\rightarrow$] Rotation starts and ends with the same set.
                    \item[$\rightarrow$] MLP outputs can't just follow the 90$\degree$ rotation!
                    \item[$\rightarrow$] There must be a \textcolor{red}{discontinuity} between \textcolor{colorTwo}{(c)} and \textcolor{colorTwo}{(d)}!\\
                    All the outputs have to jump 90$\degree$ anti-clockwise.
                \end{itemize}
            \end{itemize}

            \vspace{0.5cm}
            Conclusion:
            \begin{itemize}
                \item Smooth change of set requires discontinuous change of MLP outputs.
                \item To predict \textbf{unordered sets}, we should use an \textbf{unordered model}.
            \end{itemize}
        }












        % middle column
        \column{0.333333333333}

        % TL;DR of paper
        \useblockstyle{Emphasis}
        \block{ }{

            \color{white}
            \centering

            % massive font size
            \fontsize{70}{140}\selectfont
            To \textbf{predict a set} from a vector,
            use gradient descent to find a set
            that \textbf{encodes} to that vector.

        }
        % back to normal blocks
        \useblockstyle{Block}

        \block[bodyoffsety=-10cm, titleoffsety=-10cm]{The idea}{
            \begin{itemize}
                \item \emph{Similar} set inputs encode to \emph{similar} feature vectors.
                \item \emph{Different} set inputs encode to \emph{different} feature vectors.
            \end{itemize}
            $\rightarrow$ Minimise the difference between predicted and target set by minimising the difference between their feature vectors.

            \vspace{1cm}
            \begin{center}
                \scalebox{2.9}{
                    \begin{tikzpicture}
                       \pic {mnist1};
                    \end{tikzpicture}
                }
            \end{center}
            \vspace{1cm}
            \begin{center}
                \scalebox{2.9}{
                    \begin{tikzpicture}
                        \pic {mnist2};
                    \end{tikzpicture}
                }
            \end{center}
            \vspace{1cm}
            \begin{itemize}
                \item Train (shared) encoder weights by minimising the \textcolor{colorThree!50!black}{set loss}.
                \item Gradients of permutation-\emph{invariant} functions are \emph{equivariant}.
                \begin{itemize}
                    \item[$\rightarrow$] All gradient updates \textcolor{colorTwo}{$\partial \text{MSE} / \partial \text{set}$}  don't rely on the order of the set.
                    \item[$\rightarrow$] Our model is completely \textbf{unordered}, exactly what we wanted!
                \end{itemize}

            \end{itemize}
        }




        % right column
        \column{0.333333333333}

        \block{Bounding box set prediction}{
            \begin{center}
                \setlength{\tabcolsep}{4mm}
                \begin{tabular}{lccccc}
                    \toprule
                    Bounding box prediction & AP\textsubscript{50} & AP\textsubscript{90} & AP\textsubscript{95} & AP\textsubscript{98} & AP\textsubscript{99} \\
                    \midrule
                    MLP baseline        & 99.3\tiny$\pm$0.2 & 94.0\tiny$\pm$1.9 & 57.9\tiny$\pm$7.9 &  0.7\tiny$\pm$0.2 & 0.0\tiny$\pm$0.0 \\
                    RNN baseline        & 99.4\tiny$\pm$0.2 & 94.9\tiny$\pm$2.0 & 65.0\tiny$\pm$10.3 &  2.4\tiny$\pm$0.0 & 0.0\tiny$\pm$0.0 \\
                    \textbf{Ours} (train 10 steps, eval 10 steps) & 98.8\tiny$\pm$0.3 & 94.3\tiny$\pm$1.5 & 85.7\tiny$\pm$3.0 & \textbf{34.5}\tiny$\pm$5.7 & \textbf{2.9}\tiny$\pm$1.2 \\
                    \textbf{Ours} (train 10 steps, eval 20 steps) & \textbf{99.8}\tiny$\pm$0.0 & \textbf{98.7}\tiny$\pm$1.1 & \textbf{86.2}\tiny$\pm$7.2 & 24.3\tiny$\pm$8.0 & 1.4\tiny$\pm$0.9 \\
                    \textbf{Ours} (train 10 steps, eval 30 steps) & \textbf{99.8}\tiny$\pm$0.1 & 96.7\tiny$\pm$2.4 & 75.5\tiny$\pm$12.3 & 17.4\tiny$\pm$7.7 & 0.9\tiny$\pm$0.7 \\
                    \bottomrule
                \end{tabular}
            \end{center}
            \begin{center}
                \scalebox{2.9}{
                    \begin{tikzpicture}
                        \pic {clevr1};
                    \end{tikzpicture}
                }
            \end{center}

            \begin{itemize}
                \item Simply replace input encoder with ConvNet image encoder.
                \item Add \textcolor{colorThree!50!black}{MSE loss} to \textcolor{colorThree!50!black}{set loss} when training the encoder and ResNet weights.
                \begin{itemize}
                    \item Forces minimisation of \textcolor{colorTwo}{MSE} to converge to something sensible.
                \end{itemize}
            \end{itemize}

            \begin{center}
                \scalebox{2.9}{
                    \begin{tikzpicture}
                        \node[inner sep=0] (y0) at (0, 0) {\includegraphics[scale=0.7, trim={3.5mm 3.5mm 3.5mm 3.5mm}, clip]{resources/clevr-15-2.pdf}};
                        \node[inner sep=0] (y1) at (6, 0) {\includegraphics[scale=0.7, trim={3.5mm 3.5mm 3.5mm 3.5mm}, clip]{resources/clevr-15-6.pdf}};
                        \node[inner sep=0] (y10) at (12, 0) {\includegraphics[scale=0.7, trim={3.5mm 3.5mm 3.5mm 3.5mm}, clip]{resources/clevr-15-10.pdf}};
                        \node[inner sep=0] (y20) at (18, 0) {\includegraphics[scale=0.7, trim={3.5mm 3.5mm 3.5mm 3.5mm}, clip]{resources/clevr-15-20.pdf}};

                        \node [scale=0.2, above = 0mm of y0] {Step 2};
                        \node [scale=0.2, above = 0mm of y1] {Step 6};
                        \node [scale=0.2, above = 0mm of y10] {Step 10};
                        \node [scale=0.2, above = 0mm of y20] {Step 20};
                    \end{tikzpicture}
                }
                \scalebox{2.9}{
                    \begin{tikzpicture}
                        \node[inner sep=0] (y0) at (0, 0) {\includegraphics[scale=0.7, trim={3.5mm 3.5mm 3.5mm 3.5mm}, clip]{resources/clevr-30-2.pdf}};
                        \node[inner sep=0] (y1) at (6, 0) {\includegraphics[scale=0.7, trim={3.5mm 3.5mm 3.5mm 3.5mm}, clip]{resources/clevr-30-6.pdf}};
                        \node[inner sep=0] (y10) at (12, 0) {\includegraphics[scale=0.7, trim={3.5mm 3.5mm 3.5mm 3.5mm}, clip]{resources/clevr-30-10.pdf}};
                        \node[inner sep=0] (y20) at (18, 0) {\includegraphics[scale=0.7, trim={3.5mm 3.5mm 3.5mm 3.5mm}, clip]{resources/clevr-30-20.pdf}};
                    \end{tikzpicture}
                }
            \end{center}
        }

        \block{Object detection}{
            \begin{center}
                \setlength{\tabcolsep}{4mm}
                \begin{tabular}{lccccc}
                    \toprule
                    Object attribute prediction & AP\textsubscript{$\infty$} & AP\textsubscript{1} & AP\textsubscript{0.5} & AP\textsubscript{0.25} & AP\textsubscript{0.125}   \\
                    \midrule
                    MLP baseline & 3.6\tiny$\pm$0.5 & 1.5\tiny$\pm$0.4 & 0.8\tiny$\pm$0.3 & 0.2\tiny$\pm$0.1 & 0.0\tiny$\pm$0.0 \\
                    RNN baseline & 4.0\tiny$\pm$1.9 & 1.8\tiny$\pm$1.2 & 0.9\tiny$\pm$0.5 & 0.2\tiny$\pm$0.1 & 0.0\tiny$\pm$0.0 \\
                    \textbf{Ours} (train 10 steps, eval 10 steps) & 72.8\tiny$\pm$2.3 & 59.2\tiny$\pm$2.8 & 39.0\tiny$\pm$4.4 & 12.4\tiny$\pm$2.5 & 1.3\tiny$\pm$0.4 \\
                    \textbf{Ours} (train 10 steps, eval 20 steps) & 84.0\tiny$\pm$4.5 & 80.0\tiny$\pm$4.9 & \textbf{57.0}\tiny$\pm$12.1 & \textbf{16.6}\tiny$\pm$9.0 & \textbf{1.6}\tiny$\pm$0.9 \\
                    \textbf{Ours} (train 10 steps, eval 30 steps) & \textbf{85.2}\tiny$\pm$4.8 & \textbf{81.1}\tiny$\pm$5.2 & 47.4\tiny$\pm$17.6 & 10.8\tiny$\pm$9.0 & 0.6\tiny$\pm$0.7 \\
                    \bottomrule
                \end{tabular}
            \end{center}


            \vspace{0.5cm}
            \begin{minipage}{0.75\linewidth}
                \scriptsize
                \setlength{\tabcolsep}{1.4mm}
                \begin{tabular}{ccccc}
                    \normalsize Input & \normalsize Step 5 & \normalsize Step 10 & \normalsize Step 20 & \normalsize Target\\
                    \midrule
                    \multirow{4}{*}{\includegraphics[width=0.36\linewidth, trim={0 0 8mm 0}, clip]{resources/img-0}}
                    & x, y, z = (-0.14, 1.16, 3.57) & x, y, z = (-2.33, -2.41, 0.73) & x, y, z = (-2.33, -2.42, 0.78) & x, y, z = (-2.42, -2.40, 0.70) \\
                    & large \textcolor{red}{purple} \textcolor{red}{rubber} \textcolor{red}{sphere} & large yellow metal cube & large yellow metal cube & large yellow metal cube \\\\
                    & x, y, z = (0.01, 0.12, 3.42) & x, y, z = (-1.20, 1.27, 0.67) & x, y, z = (-1.21, 1.20, 0.65) & x, y, z = (-1.18, 1.25, 0.70)\\
                    & large \textcolor{red}{gray} \textcolor{red}{metal} \textcolor{red}{cube} & large purple rubber sphere & large purple rubber sphere & large purple rubber sphere \\\\
                    & x, y, z = (0.67, 0.65, 3.38) & x, y, z = (-0.96, 2.54, 0.36) & x, y, z = (-0.96, 2.59, 0.36) & x, y, z = (-1.02, 2.61, 0.35)\\
                    & small \textcolor{red}{purple} \textcolor{red}{metal} \textcolor{red}{cube} & small gray rubber sphere & small gray rubber sphere & small gray rubber sphere \\\\
                    & x, y, z = (0.67, 1.14, 2.96) & x, y, z = (1.61, 1.57, 0.36) & x, y, z = (1.58, 1.62, 0.38) & x, y, z = (1.74, 1.53, 0.35)\\
                    & small purple \textcolor{red}{rubber} \textcolor{red}{sphere} & small \textcolor{red}{yellow} metal cube & small purple metal cube & small purple metal cube \\
                \end{tabular}

            \end{minipage}

        }


    \end{columns}

    % QR code
    \node [fill=white, inner sep=0] at (\paperwidth/2 - \paperwidth/3 - 4cm + 1.05cm, 8cm) {\textcolor{titlebgcolor}{\qrcode[height=3.07cm, padding]{https://github.com/Cyanogenoid/dspn}}};
    \node [color=white, align=right] at (\paperwidth/2 - \paperwidth/3 - 16cm + 7mm, 7.33cm) {\large Code and pre-trained models available at\\\url{https://github.com/Cyanogenoid/dspn}};


    % light blue bar behind the entire middle column to highlight it
    \begin{scope}[on background layer]
        \draw [fill=blocktitlebgcolor] (-\paperwidth/2 + \paperwidth/3, -\paperheight/2) rectangle ++(\paperwidth/3, \paperheight);
    \end{scope}

    % footer
    % adjust height to fill out empty space at bottom to look nice
    \draw [fill=titlebgcolor] (-\paperwidth/2, -\paperheight/2) rectangle ++(\paperwidth, 1.7cm);

\end{document}
