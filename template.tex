%!TEX root = poster.tex

\tikzposterlatexaffectionproofoff
\usetheme{Basic}

% colours are picked from the southampton.ac.uk weebsite
\definecolorstyle{Soton} {
    \definecolor{colorOne}{HTML}{002e3b}    % dark blue-ish
    \definecolor{colorTwo}{HTML}{005c84}    % blue-ish
    \definecolor{colorThree}{HTML}{fcbc00}  % orange-ish
}{
    % Background Colors
    \colorlet{backgroundcolor}{white}
    \colorlet{framecolor}{colorOne}
    % Title Colors
    \colorlet{titlefgcolor}{white}
    \colorlet{titlebgcolor}{colorOne}
    % Block Colors
    \colorlet{blocktitlebgcolor}{colorTwo}
    \colorlet{blocktitlefgcolor}{white}
    \colorlet{blockbodybgcolor}{white}
    \colorlet{blockbodyfgcolor}{black}
    % Innerblock Colors
    \colorlet{innerblocktitlebgcolor}{white}
    \colorlet{innerblocktitlefgcolor}{black}
    \colorlet{innerblockbodybgcolor}{colorTwo!20!white}
    \colorlet{innerblockbodyfgcolor}{black}
    % Note colors
    \colorlet{notefgcolor}{black}
    \colorlet{notebgcolor}{colorTwo!50!white}
    \colorlet{noteframecolor}{colorTwo}
}

% style the default blocks
\defineblockstyle{Block}{
    titlewidthscale=1, bodywidthscale=1, titleleft,
    titleoffsetx=0pt, titleoffsety=0pt, bodyoffsetx=0pt, bodyoffsety=0pt,
    bodyverticalshift=0pt, roundedcorners=0, linewidth=0pt, titleinnersep=1cm,
    bodyinnersep=1cm
}{
    \ifBlockHasTitle%
        \draw[draw=none, fill=blocktitlebgcolor]
           (blocktitle.south west) rectangle (blocktitle.north east);
    \fi%
    \draw[draw=none, fill=blockbodybgcolor] %
        (blockbody.north west) [rounded corners=0] -- (blockbody.south west) --
        (blockbody.south east) [rounded corners=0]-- (blockbody.north east) -- cycle;
}

% style the TL;DR block
\defineblockstyle{Emphasis}{
    titlewidthscale=1, bodywidthscale=1, titleleft,
    bodyverticalshift=0pt, roundedcorners=0, linewidth=0pt, titleinnersep=1cm,
    bodyinnersep=1cm
}{
    \ifBlockHasTitle%
        \draw[draw=colorTwo, fill=blocktitlebgcolor]
           (blocktitle.south west) rectangle (blocktitle.north east);
    \fi%
    \draw[draw=none, fill=colorTwo] %
        (blockbody.north west) [rounded corners=0] -- (blockbody.south west) --
        (blockbody.south east) [rounded corners=0]-- (blockbody.north east) -- cycle;
}

% very simple title style
\definetitlestyle{Simple}{
    width=\paperwidth,
    titletotopverticalspace=0mm,
    titletoblockverticalspace=20mm,
    innersep=1cm,
}{
    \draw[draw=none, fill=titlebgcolor]%
    (\titleposleft,\titleposbottom) rectangle (\titleposright,\titlepostop); %
}

\usecolorstyle{Soton}
\usetitlestyle{Simple}
\useblockstyle{Block}
\useinnerblockstyle[roundedcorners=0]{Default}

% uncomment for long titles
% % https://tex.stackexchange.com/questions/180234/how-can-i-make-my-title-wrap-in-a-tikzposter
% \makeatletter
% \def\title#1{\gdef\@title{\scalebox{\TP@titletextscale}{%
%     \begin{minipage}[t]{0.4\linewidth}
%         #1
%         \par
%         \end{minipage}%
%     }}}
% \makeatother

% fix tikzlibrary{backgrounds}: https://tex.stackexchange.com/questions/236645/tikzposter-does-not-work-with-pgfplots/236672
\pgfdeclarelayer{background}
\pgfsetlayers{backgroundlayer,background,main,notelayer}


% footnotes: https://tex.stackexchange.com/questions/350010/how-to-write-a-footnote-in-tikz-poster-class/350018
\let\thempfootnote\thefootnote%http://tex.stackexchange.com/questions/956/footnotemark-and-footnotetext-in-minipage#959
\newcommand\printfootnote[1]{% to get different numbers for different footnotes
\addtocounter{footnote}{1}%
\footnotetext{#1}}
